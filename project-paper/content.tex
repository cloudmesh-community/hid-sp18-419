\title{Benchmarking Hadoop and Spark}


\author{Min Chen}
\affiliation{%
  \institution{Indiana University}
  \streetaddress{School of Informatics, Computing, and Engineering}
  \city{Bloomington}
  \state{IN}
  \postcode{47408}
}
\email{mc43@iu.edu}

\author{Gregor von Laszewski}
\affiliation{%
  \institution{Indiana University}
  \streetaddress{Smith Research Center}
  \city{Bloomington}
  \state{IN}
  \postcode{47408}
}
\email{laszewski@gmail.com}

\author{Bertolt Sobolik}
\affiliation{%
  \institution{Indiana University}
  \streetaddress{School of Informatics, Computing, and Engineering}
  \city{Bloomington}
  \state{IN}
  \postcode{47408}
}
\email{bsobolik@iu.edu}


% The default list of authors is too long for headers}
\renewcommand{\shortauthors}{M. Chen, G. v. Laszewski, B. Sobolik}


\begin{abstract}
\TODO{Add model of Raspberry Pi (3B?)}
\TODO{Add data set used}
Hadoop clusters were created on five networked Raspberry Pis. The aim
was to create the cluster in a repeatable and scalable fashion,
automating as much of the setup process as possible. Hadoop was
deployed and configured to run as Docker containers and directly on
the operating system. Configurations were tested to compare Docker
against direct installation. Hadoop was also configured in Docker and
natively on Futuresystems Echo and similar performance tests were done
to compare the performance of the configuration on Echo against that
of the Pis.
\end{abstract}

\keywords{Hadoop, Raspberry, Pi, Docker, Futuresystems}


\maketitle

\section{Introduction}
\TODO{Write an introduction.}


\section{Technology Used}
\TODO{Describe Raspberry Pis.}
\TODO{Describe Echo.}
\TODO{Describe Hadoop.}
Docker is a technology that allows applications to run in containers
instead of full VMs. It leverages control groups and name space
isolation in the Linux kernel. Containers start up much faster than
VMs and is supported by all the major public cloud
vendors~\cite{Foster:2017:CCS:3158276}. The most current stable
release of Docker Community Edition was used for this project
(18.03.0-ce).


\section{Deployment}
To configure the Raspberry Pis, first SD cards were burned using a
script that takes start and end machine id numbers as an input. The
option to perform DHCP setup can be enabled with a flag. If DHCP is
not used, the script assigns static IPs to each machine.

\section{Data}
\TODO{Describe data}


\section{Benchmarking Process}
\TODO{Describe MapReduce task used for benchmarking.}


\section{Results}
\TODO{Add results.}

\begin{table}[hbt]
\centering
\caption{Benchmarking results}\label{t:results-table}
\begin{tabular}{llll}
Platform    & Docker & Deployment time & MapReduce Time \\
Pi 5 nodes  & Yes    & TBD             & TBD            \\
Pi 5 nodes  & No     & TBD             & TBD            \\
Echo        & Yes    & TBD             & TBD            \\
Echo        & No     & TBD             & TBD            \\
\end{tabular}
\end{table}



\section{Conclusion}

\TODO{Put here an conclusion. Conclusion and abstracts must not have any
citations in the section.}


\begin{acks}

  The authors would like to thank Dr.~Gregor~von~Laszewski for his
  support and suggestions to write this paper.

\end{acks}

\bibliographystyle{ACM-Reference-Format}
\bibliography{report}
